% !TEX encoding = UTF-8
% !TEX TS-program = pdflatex
% !TEX root = ../tesi.tex

%**************************************************************
\chapter{Valutazioni finali}
\label{cap:valutazioni-finali}
%**************************************************************
\section{Soddisfacimento Obiettivi}
Al termine del periodo di stage, gli obiettivi prefissati nel piano di lavoro possono considerarsi completamente soddisfatti.
In particolare, facendo riferimento alla nomenclatura riportata nella sezione \hyperref[subsec:obiettivi]{Obiettivi}, troviamo:
\begin{itemize}
	\item \underline{\textit{MIN01}}: lo studio degli strumenti è stato effettuato, discusso e infine documentato su Confluence;
	\item \underline{\textit{MIN02,MAX01}}: l'infrastruttura è stata realizzata negli ambienti indicati;
	\item \underline{\textit{MIN03}}: la stesura della suite di test di carico è avvenuta tramite JMeter mentre l'installazione della stessa si è verificate tramite l'utilizzo della \gls{gui} sviluppata;
	\item \underline{\textit{MIN04}}: la documentazione relativa a progettazione, sviluppo e utilizzo del prodotto è stata archiviata su Confluence;
	\item \underline{\textit{MAX02}}: i test sono stati eseguiti sull'infrastruttura di accettazione/collaudo;
	\item \underline{\textit{MAX03}}: i risultati dei test di carico sono stati recapitati via E-Mail e successivamente visionati dal team Legalmail.
\end{itemize}
Inoltre, grazie all'ottima collaborazione tra i team \textit{Legalmail}, \textit{Infrastrutture} e \textit{Network}, i test sono stati eseguiti anche in un ambiente esterno, AWS, non previsto dal piano di lavoro iniziale.
\section{Maturazione Professionale}
Sono complessivamente molto soddisfatto dell'esperienza di stage svolta in Infocert S.p.A: oltre all'ottimo ambiente di lavoro riscontrato in azienda, ho avuto modo di accrescere la mia figura professionale. \\
Alla luce delle conoscenze e competenze acquisite durante il tirocinio, elencate qui di seguito, posso affermare con sicurezza che le mie aspettative personali sono state completamente corrisposte. 
\subsection{Enterprise}
Lavorando in Infocert ho avuto modo di vedere come sono strutturate le aziende di stampo Enterprise, di capire come è organizzato il lavoro in più team adempienti a funzionalità differenti e soprattutto di imparare a collaborare non solo con il proprio gruppo, ma anche con componenti di team diversi ponendo come ragione comune l'obiettivo da raggiungere.
\subsection{SCRUM}
Grazie alla disponibilità dei componenti del team Legalmail, ho avuto modo di partecipare in modo attivo (come per\textit{Daily Scrum} e \textit{Sprint Review}) e passivo (\textit{Sprint Planning}, e \textit{Retrospective}) alle cerimonie SCRUM indette dal team, potendo vedere all'opera un'implementazione concreta delle metodologie Agile, spesso affrontate in modo troppo astratto durante il percorso accademico.
\subsection{Strumenti di test di carico}
Grazie allo studio eseguito all'inizio del tirocinio ho avuto modo di ampliare il mio bagaglio teorico sugli strumenti di test di carico. Pur non avendoli utilizzati nel concreto ho comunque realizzato una comparativa dei loro pregi e difetti, permettendomi in futuro di possedere già una base di conoscenze per scegliere il miglior strumento adatto al caso d'uso aziendale.
\subsection{Infrastructure as Code}
Grazie alla seconda parte dello studio sono riuscito ad osservare più da vicino il mondo dell'IaC, metodologia che mi ha sempre affascinato ma che non ero riuscito ad approfondire in precedenza per mancanza di tempo. I concetti appresi durante il tirocinio mi permetteranno di osservare con più cognizione tutti i termini e le pratiche molto usati oggi nel mondo dell'ingegneria del software. 
\subsection{JMeter}
Dovendo progettare, implementare e eseguire una suite di test di carico ho sviluppato le mie competenze con JMeter, strumento scelto per l'esecuzione dei test di carico.
\subsection{Terraform}
Grazie all'integrazione con i servizi di AWS ho imparato ad utilizzare, seppur nella sua forma più basilare, il provisioner Terraform, aggiungendo un'arma in più al mio arsenale di competenze per una possibile carriera da DevOps.
\subsection{Python}
Il miglioramento più grande l'ho avuto sicuramente con l'utilizzo del linguaggio di programmazione Python. In passato avevo già avuto modo di utilizzarlo ma solo per la realizzazione di piccoli script o programmi di poco conto. Lo sviluppo di un intero progetto e l'integrazione con il \gls{framework} Flask mi hanno permesso di approfondirne aspetti meno superficiali e aumentare la mia sicurezza durante il suo utilizzo. 
\section{Distanza Università-Lavoro}
%Agile troppo astratto, tecnologie open apprese solo tardi, maggior parte comunque coperto in SWE, comunque sia basi veramente solide, università sperimenta lavoro meglio andare sul sicuro.