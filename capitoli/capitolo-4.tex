% !TEX encoding = UTF-8
% !TEX TS-program = pdflatex
% !TEX root = ../tesi.tex

%**************************************************************
\chapter{Valutazioni finali}
\label{cap:valutazioni-finali}
%**************************************************************
\section{Soddisfacimento Obiettivi}
Al termine del periodo di stage, gli obiettivi prefissati nel piano di lavoro possono considerarsi completamente soddisfatti.
In particolare, facendo riferimento alla nomenclatura riportata nella sezione \hyperref[subsec:obiettivi]{Obiettivi}, troviamo:
\begin{itemize}
	\item \underline{\textit{MIN01}}: lo studio degli strumenti è stato effettuato, discusso e infine documentato su Confluence;
	\item \underline{\textit{MIN02,MAX01}}: l'infrastruttura è stata realizzata negli ambienti indicati;
	\item \underline{\textit{MIN03}}: la stesura della suite di test di carico è avvenuta tramite JMeter mentre l'installazione della stessa si è verificate tramite l'utilizzo della \gls{gui} sviluppata;
	\item \underline{\textit{MIN04}}: la documentazione relativa a progettazione, sviluppo e utilizzo del prodotto è stata archiviata su Confluence;
	\item \underline{\textit{MAX02}}: i test sono stati eseguiti sull'infrastruttura di accettazione/collaudo;
	\item \underline{\textit{MAX03}}: i risultati dei test di carico sono stati recapitati via E-Mail e successivamente visionati dal team Legalmail.
\end{itemize}
Inoltre, grazie all'ottima collaborazione tra i team \textit{Legalmail}, \textit{Infrastrutture} e \textit{Network}, i test sono stati eseguiti anche in un ambiente esterno, AWS, non previsto dal piano di lavoro iniziale.
\section{Consuntivo Temporale}
Nonostante gli obiettivi siano stati completamente soddisfatti, il piano di lavoro temporale non è stato seguito in modo completamente lineare: il primo mese, rappresentante i primi due sprint, ha rispettato nella sua totalità le milestones prefissate, mentre il secondo è risultato un po' più turbolento a causa delle nuove esigenze nate durante progettazione e sviluppo dell'applicativo.
In particolare, facendo riferimento alla nomenclatura presentata nella sezione \hyperref[subsec:pianificazione]{Pianificazione Temporale}: 
\begin{itemize}
	\item \underline{\textit{M1}}: al termine del primo sprint lo studio degli strumenti necessari è stato portato a termine, consolidando i concetti teorici necessari per la progettazione;
	\item \underline{\textit{M2}}: al termine del secondo sprint una prima versione, limitata, dell'orchestratore era stata progettata e sviluppata, permettendone un'accesso via \gls{cli}. Un primo giro di \gls{refactor} è stato effettuato e le prime configurazioni Puppet sono state caricate nell'ambiente di sviluppo;
	\item \underline{\textit{M3}}: il terzo sprint, dedicato alla predisposizione degli ambienti di test e accettazione, ha subito delle variazioni: le nuove esigenze, come lo sviluppo della \gls{gui}, hanno richiesto un altro sforzo di progettazione e sviluppo, proponendo alla fine della sesta settimana un applicativo più completo e funzionale contrapposto alla mancanza d'installazione all'interno dell'ambiente d'accettazione;
	\item \underline{\textit{M4, M5}}: l'ultimo sprint è stato quello più turbolento: nonostante dovesse essere interamente dedicato all'esecuzione dei test di carico e relativa documentazione, le difficoltà incontrate nello sviluppo di una \gls{gui} efficiente e la possibilità di attivare l'ambiente AWS hanno cambiato la programmazione effettuata a monte. La settima settimana è stata quindi dedicata alla sistemazione dell'interfaccia grafica e all'attivazione dell'ambiente nel \gls{cloud}, con qualche accenno alla documentazione. L'ottava invece ha riposto completa attenzione alla predisposizione dell'ambiente d'accettazione, l'esecuzione dei test di carico e la stesura della documentazione su Confluence.
\end{itemize}
Il quarto sprint si è poi concluso con la presentazione e dimostrazione del lavoro svolto al team Legalmail.
\section{Maturazione Professionale}
Sono complessivamente molto soddisfatto dell'esperienza di stage svolta in Infocert S.p.A: oltre all'ottimo ambiente di lavoro riscontrato in azienda, ho avuto modo di accrescere la mia figura professionale. \\
Alla luce delle conoscenze e competenze acquisite durante il tirocinio, elencate qui di seguito, posso affermare con sicurezza che le mie aspettative personali sono state completamente corrisposte. 
\subsection{Enterprise}
Lavorando in Infocert ho avuto modo di vedere come sono strutturate le aziende di stampo Enterprise, di capire come è organizzato il lavoro in più team adempienti a funzionalità differenti e soprattutto di imparare a collaborare non solo con il proprio gruppo, ma anche con componenti di team diversi ponendo come ragione comune l'obiettivo da raggiungere.
\subsection{SCRUM}
Grazie alla disponibilità dei componenti del team Legalmail, ho avuto modo di partecipare in modo attivo (come per\textit{Daily Scrum} e \textit{Sprint Review}) e passivo (\textit{Sprint Planning}, e \textit{Retrospective}) alle cerimonie SCRUM indette dal team, potendo vedere all'opera un'implementazione concreta delle metodologie \gls{agile}, spesso affrontate in modo troppo astratto durante il percorso accademico.\\
Un altro aspetto delle metodologie \gls{agile} l'ho assorbito nel secondo mese di stage: come specificato nel consuntivo temporale ho dovuto infatti reagire in modo elastico alle nuove esigenze del progetto, non preventivate inizialmente nel piano di lavoro. Questa reattività ai cambiamenti è essenziale all'interno dei modelli di sviluppo aziendali e la sua concreta consapevolezza mi sarà d'aiuto in potenziali situazioni lavorative future.
\subsection{Strumenti di test di carico}
Grazie allo studio eseguito all'inizio del tirocinio ho avuto modo di ampliare il mio bagaglio teorico sugli strumenti di test di carico. Pur non avendoli utilizzati nel concreto ho comunque realizzato una comparativa dei loro pregi e difetti, permettendomi in futuro di possedere già una base di conoscenze per scegliere il miglior strumento adatto al caso d'uso aziendale.
\subsection{Infrastructure as Code}
Grazie alla seconda parte dello studio sono riuscito ad osservare più da vicino il mondo dell'IaC, metodologia che mi ha sempre affascinato ma che non ero riuscito ad approfondire in precedenza per mancanza di tempo. I concetti appresi durante il tirocinio mi permetteranno di osservare con più cognizione tutti i termini e le pratiche molto usati oggi nel mondo dell'ingegneria del software. 
\subsection{JMeter}
Dovendo progettare, implementare e eseguire una suite di test di carico ho sviluppato le mie competenze con JMeter, strumento scelto per l'esecuzione dei test di carico.
\subsection{Terraform}
Grazie all'integrazione con i servizi di AWS ho imparato ad utilizzare, seppur nella sua forma più basilare, il provisioner Terraform, aggiungendo un'arma in più al mio arsenale di competenze per una possibile carriera da DevOps.
\subsection{Python}
Il miglioramento più grande l'ho avuto sicuramente con l'utilizzo del linguaggio di programmazione Python. In passato avevo già avuto modo di utilizzarlo ma solo per la realizzazione di piccoli script o programmi di poco conto. Lo sviluppo di un intero progetto e l'integrazione con il \gls{framework} Flask mi hanno permesso di approfondirne aspetti meno superficiali e aumentare la mia sicurezza durante il suo utilizzo. 
\section{Distanza Università-Lavoro}
Al termine di questo percorso di stage non ho notato grosse lacune tra le competenze richieste e quelle acquisite nel corso di studi. Sicuramente alcune competenze specifiche, come l'uso di JMeter, non sono state affrontate all'interno del percorso accademico, ma le buone basi teoriche e metodologiche fornite dalla facoltà hanno permesso di colmare in autonomia le eventuali mancanze.\\
In particolare ritengo che la presenza di progetti individuali agli inizi del corso sia molto importante, in quanto permette di accrescere la propria capacità di autoapprendimento, essenziale nel mondo del lavoro. Inoltre molte delle conoscenze teoriche necessarie per la carriera professionale vengono erogate dal corso di Ingegneria del Software, che riesce a presentare a trecentosessanta gradi il mondo dell'Informatica. Per assurdo, però, molti concetti e strumenti presentati nel corso sarebbero molto utili per gli anni precedenti, come ad esempio strumenti per controllo di versione e di gestione di progetto. \\
L'unica pecca l'ho trovata nella comprensione dei modelli Agile, a lezione infatti questi vengono presentati in maniera molto teorica non permettendo di capirne la piena essenza. In particolare relegare l'implementazione di questi modelli al gruppo di progetto, riferendomi al corso di Ingegneria del Software, risulta un esperimento un po' fine a se stesso, in quanto gli studenti non possono applicarne le metodologie a pieno, ricordando che nel mondo del lavoro le pratiche sono coadiuvate da figure professionali dedicate, come \textit{Scrum Master} e \textit{Lean Manager}. Tuttavia mi rendo conto della difficoltà di praticare queste metodologie in classe.\\
In definitiva ho trovato il corso di studi molto completo e stimolante, i vari corsi dispensati nei tre anni consentono di ottenere una solida base teorica del mondo dell'Informatica, mentre i progetti permettono di fare dell'esperienza pratica, citando in particolare il progetto di Ingegneria del Software che a mio parere rimane l'esperienza più completa e formativa del corso accademico.\\
In ultimo mi sento di fare una riflessione personale sulla differenza tra Università e Lavoro: la prima permette e stimola la sperimentazione a tutto tondo: i progetti, seppur vincolati, permettono di approcciarsi a tecnologie nuove e magari poco consolidate, permettendo di far leva sul proprio spirito di esplorazione che, a mio parere, è la parte più bella di questa disciplina. In azienda invece, seppur la sperimentazione è concessa, spesso gli studi si riconducono all'utilizzo di tecnologie ben consolidate e con poco ancora da scoprire, sacrificando lo spirito d'esplorazione a favore, come è giusto che sia, dei risultati. 