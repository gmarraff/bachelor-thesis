% !TEX encoding = UTF-8
% !TEX TS-program = pdflatex
% !TEX root = ../../tesi.tex
Fin dalla pubblicazione dei progetti proposti dalle aziende ospitanti, il mio entusiasmo per l'evento \textit{Stage-It} è cresciuto esponenzialmente. La varietà delle aziende coinvolte e la moltitudine di progetti presentati dalle stesse fanno dell'evento una grandissima opportunità offerta dal nostro corso di studi. Non capita spesso che ad un laureando venga data la possibilità di \textbf{scegliere} dove confluire all'interno del mercato del lavoro permettendogli di perseguire le proprie ambizioni professionali.\\
Avendo lavorato come sviluppatore prima e durante il percorso universitario, e avendo colto le varie sfaccettature del mondo dello sviluppo software grazie al corso di Ingegneria del Software, avevo ben chiare le prospettive che questo tirocinio avrebbe dovuto offrirmi e gli ambiti, invece, che avrei preferito evitare. \\
In primo luogo c'era la volontà di lavorare in un'azienda fortemente strutturata, con metodologie Agile ben consolidate e approcci moderni all'ingegnerizzazione del software, come l'\gls{integrazione-continua} e la filosofia \gls{devops}.\\
In secondo luogo c'era l'idea di allontanarsi un po' dalla figura dello sviluppatore, a mio parere a volte un po' troppo schiava delle esigenze, legittime e non, del cliente. L'intenzione era quella di avvicinarsi alla figura del \gls{devops} per automatizzare tutti i compiti ripetitivi e creare l'involucro che avrebbe ospitato l'applicazione vera e propria, garantendo qualità e rapidità.\\
Un'altra figura che mi sarebbe piaciuto affrontare era quella del \gls{qa}, ponendomi dalla parte opposta dello sviluppatore per scovare errori e migliorare la qualità del prodotto. \\
Dopo la partecipazione a \textit{Stage-It} tre aziende hanno catturato maggiormente la mia attenzione: 
\begin{itemize}
	\item \textbf{Infocert}: il cui progetto è ben spiegato nelle sezioni precedenti;
	\item \textbf{Finantix}\footcite{site:finantix}: che proponeva una sensazionale filosofia \gls{devops} ben rodata e integrata nella metodologia aziendale, offrendo la possibilità di ampliarne le procedure;
	\item \textbf{Thron}\footcite{site:thron}: che offriva l'inserimento in un team di \gls{qa} per migliorarne i processi.
\end{itemize}
Sebbene tutte e tre le aziende mi avessero lasciato una grande impressione, l'unicità del progetto di Infocert (non sono tante le aziende che permettono di concentrarti sui test di carico) permetteva di esplorare sia l'approccio \gls{devops}, tramite l'automazione dell'infrastruttura, e il mondo della \gls{qa}, tramite l'esecuzione dei test.
