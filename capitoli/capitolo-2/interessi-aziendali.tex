% !TEX encoding = UTF-8
% !TEX TS-program = pdflatex
% !TEX root = ../../tesi.tex
\subsection{Stage nella strategia aziendale}
Come sottolineato nella sezione \hyperref[sec:innovazione]{Propensione all'innovazione}, la collaborazione con le università svolge un ruolo cardine per la ricerca e lo sviluppo di soluzioni software all'avanguardia. \\
Grazie alla collaborazione con Università e Centri di Ricerca (Università degli studi di Padova, Politecnico di Milano, EXO Organismo di ricerca, Università di Tor Vergata, Università di Salerno e SDA Bocconi) Infocert riesce a proporre soluzioni sempre migliori a propri clienti, permettendole di sperimentare tecnologie nuove tramite le eccellenze prodotte dal sistema accademico italiano. \\
Tramite questi stage inoltre l'azienda è in grado di avvicinare e formare nuovi possibili lavoratori, offrendo in cambio esperienza diretta in un ambiente produttivo e professionale. \\
Gli stage offerti agli studenti dell'Università degli studi di Padova, ad esempio, sono finalizzati all'assunzione nel prossimo periodo ma senza obblighi contrattuali, permettendo al tirocinante di ricavare il meglio dall'esperienza offerta e all'azienda di ampliare il proprio know how tecnologico.  
\subsection{I test di carico nel ciclo di vita del software}
Infocert tramite questo progetto formativo mira ad impostare le basi per l'inserimento in modo costante dei test di carico nel ciclo di vita del software da loro rilasciato. \\
Il team Legalmail già in passato ha effettuato test di questo tipo ma, a causa di stretti vincoli temporali e dall'impossibilità di posizionare risorse esclusivamente su questo progetto, non ha avuto modo di creare un'infrastruttura che rendesse questo processo ripetibile e semplice da implementare.  
Il gruppo infatti ha da tempo abbracciato la filosofia \gls{devops}, sviluppando \gls{framework} interni ottimizzati per aderire alla pratica dell'\gls{integrazione-continua}, proprio per questo motivo il team ha estremo interesse nello sviluppare un framework facilmente integrabile nella loro piattaforma anche per l'esecuzione dei test di carico. \\
Per l'azienda quindi, l'inserimento di una risorsa che si occupi di questo progetto a tempo pieno, risulta una scelta molto efficace. \\
L'esecuzione dei test di carico, inoltre, presenta vantaggi non indifferenti e ha valenza informativa non solo per l'area tecnica ma anche per gli \gls{stakeholders}: la prima infatti può analizzare i dati ottenuti per pianificare l'ampliamento o la riduzione della flotta dei server, mentre i secondi possono utilizzare i risultati dei test come certificazione della qualità del prodotto.