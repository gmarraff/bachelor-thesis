% !TEX encoding = UTF-8
% !TEX TS-program = pdflatex
% !TEX root = ../../tesi.tex
\subsubsection{Catchers}
Per la gestione degli errori sono stati implementati 3 \textit{catchers}:
\begin{itemize}
	\item \textbf{StandardCatcher}: propaga l'errore senza catturarlo, utile per relegare la gestione dell'errore all'entità software che sta utilizzando JMeterOrchestrator;
	\item \textbf{CliCatcher}: stampa l'errore e il relativo \gls{traceback} nel dispositivo di output specificato dall'esecutore, utile per esecuzioni via \gls{cli};
	\item \textbf{MailCatcher}: notifica l'errore via E-Mail utilizzando il protocollo \gls{smtp}, utile per esecuzione in background del programma.
\end{itemize}
\subsubsection{Consumers}
Considerati i requisiti del team Legalmail è stato implementato un solo \textit{consumer}, denominato \textbf{StandardConsumer} responsabile di notificare via E-Mail il successo o il fallimento dell'esecuzione del test. In caso di successo il componente invia, sempre tramite \gls{smtp}, i risultati prodotti da JMeter. 
\subsubsection{Dealers}
Per la distribuzione dei files di input alle macchine slaves è stata implementata una sola modalità, via \gls{scp}, incapsulata nella classe \textbf{StandardDealer}.
\subsubsection{Parsers}
Per la modifica del file \texttt{jmx} è stata implementata una sola classe, \textbf{StandardParser}, che permette la modifica della configurazione di carico e di alcune parametri HTTP. I file forniti come input alla classe vengono sovrascritti.
\subsubsection{Provisioners}
Per il provisioning delle macchine sono stati sviluppati tre diversi tipi di \textit{provisioner}, in modo da soddisfare le due modalità d'esecuzione previste dal piano di lavoro:
\begin{itemize}
	\item \textbf{StandardProvisioner}: presenta un implementazione di facciata: non esegue operazioni sulle macchine ma funziona da intermediario tra l'utente e l'orchestratore, ricevendo gli ip delle macchine slaves come input e consegnandoli come output;
	\item \textbf{EnsureProvisioner}: tramite l'esecuzione di comandi immessi dall'utente, sfruttando il protocollo \gls{ssh}, si assicura che i server RMI delle macchine slaves siano attivi. Durante la fase di cleanup, la classe s'impegna a spegnere i server RMI. Necessita che le macchine slaves siano già create;
	\item \textbf{TerraformAwsProvisioner}: sfruttando le funzionalità di Terraform crea e configura per l'esecuzione dei test le macchine slaves ed eventualmente master sui \gls{datacenter} AWS. Durante la fase di cleanup utilizza nuovamente Terraform per distruggere le macchine create.
\end{itemize}
\subsubsection{Runners}
Per l'esecuzione di JMeter è stata creata una sola classe, denominata \textbf{StandardRunner}, che prevede l'esecuzione del software sia che la macchina master esista in locale (sulla stessa macchina in cui è installato l'orchestratore), sia che esista in remoto (un altro computer di cui si conosce l'identificativo). Questa classe oltre che dell'esecuzione si preoccupa di notificarne all'orchestratore il successo/fallimento e il percorso dove sono stati generati i files di risultato.
\subsubsection{Wizards}
Per le attuali esigenze è stato generato un solo \textit{wizard}: \textbf{YamlCliWizard}. Questa \textit{actions} svolge un ruolo cruciale per l'usabilità dell'applicativo, permettendo di specificare le configurazioni in due diverse risorse: un file in formato \gls{yaml} e i parametri del costruttore di \textit{JmeterOrchestrator}. \\
Questa distinzione è stata fatta per specificare le configurazioni di default e probabilmente immutabili, come le configurazioni \gls{smtp} o \gls{ssh}, all'interno del file \gls{yaml}, mentre i parametri più variabili, come la configurazione di carico o i percorsi dei file di input da utilizzare, vengono comunicati all'orchestratore al momento dell'esecuzione, in modo da essere facilmente adattati allo scenario di test che si vuole utilizzare. \\
Tramite questo \textit{wizard} è possibile inoltre iniettare le \textit{actions} desiderate specificandole nel file \gls{yaml} o nei parametri utente.