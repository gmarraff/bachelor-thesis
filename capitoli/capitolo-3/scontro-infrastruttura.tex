% !TEX encoding = UTF-8
% !TEX TS-program = pdflatex
% !TEX root = ../../tesi.tex
\subsection{Provisioning nell'intranet}
La soluzione proposta discussa nella sezione precedente, seppur condivisa dal tutor aziendale, ha dovuto subire degli accorgimenti per aderire agli standard aziendali.\\
Come spiegato nella sezione \hyperref[subsec:infra]{Infrastruttura e Applicazioni}, il provisioning delle macchine virtuali all'interno dell'\gls{intranet} deve rispettare gli standard definiti dal team \textit{Infrastructure}, che attualmente prevede l'instanziazione dei server solo tramite Foreman. Inoltre le policy aziendali, allo status attuale, non permettono l'integrazione con le \gls{apig} di Foreman, rendendo troppo oneroso lo sviluppo di un automatismo di provisioning. \\
L'integrazione con i servizi di AWS inoltre, seppure già adottati in piccola parte per alcuni servizi aziendali, richiede la collaborazione sia con il team \textit{Infrastructure} che con il team \textit{Network}, responsabile della gestione dell'instrastruttura di rete.

\subsection{Piano di lavoro}
A seguito di queste considerazioni è stato approvato il progetto dell'orchestratore in python, denominato \textbf{JMeterOrchestrator}, che, oltre alla gestione delle meccaniche di JMeter, avrebbe dovuto permettere due modalità d'uso:
\begin{enumerate}
	\item \textbf{Provisioning Interno (Obbligatorio)}: utilizzando l'infrastruttura aziendale, ovvero creando a priori le istanze tramite Foreman e utilizzandole tramite l'orchestratore;
	\item \textbf{Provisioning Esterno (Opzionale)}: sfruttando terraform per creare i disposable servers su AWS direttamente dall'orchestratore.
\end{enumerate}
Il consumo dei risultati, inoltre, si sarebbe tradotto con l'invio di una E-Mail, contenente i dati prodotti, al personale del team Legalmail. \\
L'installazione di JMeterOrchestrator sarebbe comunque dovuta avvenire all'interno della rete aziendale.