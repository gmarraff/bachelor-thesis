% !TEX encoding = UTF-8
% !TEX TS-program = pdflatex
% !TEX root = ../../tesi.tex
Pur essendo molto collaborativi, sviluppatori e operatori utilizzano tecnologie diverse per soddisfare i requisiti richiesti. I primi infatti usano principalmente linguaggi di programmazione per sviluppare le nuove funzionalità, mentre i secondi adoperano anche linguaggi dichiarativi per configurare i vari ambienti di esecuzione.
\subsection{Dev}
Gli sviluppatori seguono la classica divisione in backend developers, responsabili della logica di business dell'appplicativo, e frontend developers, responsabili dell'interfacciamento tra utente e applicazione.
\subsubsection{Java 8 Enterprise Edition}
% https://stackoverflow.com/a/1197913
% https://medium.com/webbasedevelopment/why-java-is-the-most-preferred-programming-language-for-building-end-to-end-enterprise-solutions-731027246f2a
Lo standard scelto per lo sviluppo della business logic dei vari applicativi è Java\footcite{site:java}, linguaggio di programmazione orientato agli oggetti e spesso presenza indiscussa nella maggior parte degli ambienti enterprise per una moltitudine di obiettivi, tra i principali:
\begin{itemize}
	\item \textbf{Maturità}: Oracle\footcite{site:oracle} rilascia la prima versione di Java nel 1995 e conquista col tempo la maggior parte dei software sviluppati, rivelandosi fin da subito un componente robusto e sicuro per lo sviluppo di applicazioni. Le aziende di grosso taglio si sentono quindi al sicuro nell'adottare Java, assicurandosi un prodotto ben consolidato;
	\item \textbf{Curva d'apprendimento}: nei suoi principi base, Java è un linguaggio molto semplice e, essendo i team di aziende enterprise composti da molte persone, è molto facile formare nuovi sviluppatori all'utilizzo della tecnologia.\\
	Inoltre la sua diffusione ha portato negli anni alla creazione di una moltitudine di corsi ben strutturati, conquistando spazi anche nelle università;
	\item \textbf{Funzionalità}: essendo la community di sviluppatori Java molto diffusa, il mondo open source è pieno di librerie che risolvono problemi comuni a molti casi d'uso, velocizzando i tempi di sviluppo e riducendo l'introduzione di nuovi errori;
	\item \textbf{Multipiattaforma}: un applicativo Java viene scritto una sola volta e, con pochi accorgimenti, può essere distribuito su moltissime piattaforme, permettendo all'azienda di distribuire i propri prodotti senza incappare in "barriere architettoniche";
	\item \textbf{Scalabilità}: i prodotti di stampo enterprise sono utilizzati da parecchie persone, per questo l'infrastruttura che li ospita deve essere flessibile ad aumenti o diminuzioni di traffico. Java, grazie alle numerose integrazioni esistenti, si presenta come un ottimo candidato per queste necessità.
\end{itemize}
L'azienda ha sviluppato su di esso un \gls{framework} proprietario chiamato BEcon.
\subsubsection{Angular JS}
Per la parte frontend dei vari prodotti, Infocert ha sviluppato un \gls{framework} interno, FEcon, basato su AngularJS\footcite{site:angular}: un \gls{framework} strutturale per la creazione di applicazioni web dinamiche.
\subsection{Ops}
Gli operatori non hanno distinzioni interne e utilizzano Puppet per la configurazione delle macchine e Java e/o Python per la realizzazione dei test d'accettazione e sviluppo delle sonde di monitoraggio. \\
Le sonde di monitoraggio sono piccoli applicativi che verificano lo stato dei server nei vari ambienti d'esecuzione; in caso di macchine non operative avvisano i relativi responsabili.
\subsubsection{Puppet}
Puppet è uno strumento di \gls{cfg-mgmt} che permette di definire lo stato delle macchine tramite un linguaggio dichiarativo dedicato, garantendo il versionamento dell'infrastruttura e la replica delle configurazioni su più macchine con il minimo sforzo. \\
Operando le modifiche ai server esclusivamente tramite Puppet è infatti possibile mantenere consistente lo stato dell'infrastruttura, rendendo più rapida la propagazione delle modifiche in flotte composte da molti server. \\
Un altro punto di forza di Puppet è quello di definire delle classi tramite il linguaggio proprietario dell'applicativo, permettendo ai meno esperti di utilizzarle senza conoscerne l'implementazione interna (proprio come una classe vera e propria, con metodi privati e pubblici) e rendendo più accessibile la gestione della configurazione. Queste classi infatti possono essere configurate tramite file YAML, rendendone possibile l'utilizzo anche a chi non è formato per utilizzare il linguaggio interno di Puppet.
\subsubsection{Python}
Python è un linguaggio di programmazione non tipizzato che da tempo sta conquistando sempre più aree dello sviluppo software. Grazie alla sua bassa curva d'apprendimento e alle innumerevoli librerie open-source presenti sul mercato, risulta la scelta ideale per molte soluzioni software, tra le quali troviamo le utility di scripting, il calcolo distribuito, il data mining e lo sviluppo dei test di sistema. \\
La forza di python sta nel permettere la realizzazione di compiti complessi tramite la scrittura poche righe di codice, mantenendo alta la leggibilità dello stesso. \\