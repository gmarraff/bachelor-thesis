% !TEX encoding = UTF-8
% !TEX TS-program = pdflatex
% !TEX root = ../../tesi.tex
\subsection{Jira}
Jira\footcite{site:jira} è uno strumento rilasciato da Atlassian per gestire tutta la componente organizzativa del ciclo di vita del software, è ideato per supportare i modelli agili ed è lo strumento più utilizzato dalle aziende che ne adottano le metodologie. \\
Offre funzionalità di \textit{Project Managment}, \textit{Issue Tracking}, \textit{Agile Reporting} e \textit{Roadmap Planning}.
All'interno dell'azienda viene ad esempio utilizzato per la gestione delle SCRUM board e del release planning, oltre che a tutti gli aspetti della gestione di progetto.
\subsection{Confluence}
Confluence\footcite{site:confluence} è uno strumento rilasciato sempre da Atlassian per la gestione della parte documentale non per forza legata alla documentazione del software: descrizioni di procedure, guide per installazioni, inventario indirizzi macchine virtuali e tutte le informazioni necessarie per distribuire la conoscenza aziendale a tutti i dipendenti. \\
L'azienda ha adottato Confluence per racchiudere in un punto centrale tutte queste informazioni, permettendo anche di commentare i documenti scritti da altre persone in modo da migliorare attivamente la qualità della documentazione. \\
Una parte del mio progetto consisteva nello scrivere una guida su Confluence per l'instanziazione dell'infrastruttura di test ed il lancio dello stesso.
\subsection{Gitlab}
Il versionamento degli applicativi viene fatto tramite Git\footcite{site:git} e i repository sono ospitati su Gitlab\footcite{site:gitlab}: una piattaforma che permette la gestione centralizzate dei repository Git, permettendo l'amministrazione dei permessi d'accesso tramite una semplice interfaccia grafica.
Tutti gli applicativi e le configurazioni dell'infrastruttura vengono ospitati sulla piattaforma e il versionamento segue le regole del Git-Flow: le funzionalità nuove vengono sviluppate su un branch dedicato, per poi essere spostate successivamente nei \gls{branch} di sviluppo, accettazione e infine produzione. \\
Il rispetto del Git-Flow non ha solo scopo formale: la definizione corretta dei branch scatena tutto un processo che dall'esecuzione dei test effettua il \gls{deploy} dell'applicazione su Artifactory. \\
Questo procedimento prende il nome di \gls{integrazione-continua} e viene gestito da Jenkins.
\subsection{Jenkins}
Jenkins\footcite{site:jenkins} è uno strumento open source sviluppato dalla Jenkins CI community che permette l'implementazione di sistemi di \gls{integrazione-continua} su web-server Java. \\
Jenkins è molto flessibile, permette la sua esecuzione in modo automatizzato e/o manuale ed è predisposto per essere esteso tramite plugin scritti dalla community. \\
All'interno dell'azienda la sua esecuzione è innescata dai commit su gitlab e, in base al \gls{branch} di appartenenza del commit, scatena una serie di operazioni volete ad automatizzare la verifica, validazione e distribuzione su Artifactory del software.
\subsection{Artifactory}
Artifactory\footcite{site:artifactory} è uno strumento di repository per pacchetti applicativi (binaries) sviluppato da JFrog, questo permette il versionamento delle \gls{build} e una conseguente diminuzione dei tempi di \gls{deploy} dell'applicativo.
All'interno dell'azienda, Jenkins effettua la \gls{build} dei codici sorgenti e li carica su Artifactory; questi pacchetti, stabili e versionati, vengono poi indirizzati dai sistemi di \gls{cfg-mgmt} per essere installati sui server operativi.
\subsection{Puppet Enterprise}
Puppet Enterprise\footcite{site:puppet} è la versione Enterprise dello strumento di \gls{cfg-mgmt} Puppet, questa piattaforma offre una semplice gestione della configurazione per grandi flotte di server in modo completamente automatizzato e parallelo, con la possibilità di gestire la propagazione delle modifiche tramite un'intuitiva interfaccia grafica.\\
All'interno dell'azienda lo strumento viene utilizzato per applicare le modifiche ai server in modo rapido e affidabile.
\subsection{Foreman}
Foreman\footcite{site:foreman} è uno strumento per la completa gestione del ciclo di vita delle macchine fisiche e/o virtuali. Si presente come un'interfaccia grafica e permette l'approvvigionamento dei server sia su \gls{datacenter} privati che soluzioni nel cloud. \\
All'interno dell'azienda il team infrastrutture crea dei modelli di macchine interfacciandosi al \gls{datacenter} privato, mentre i team di sviluppo personalizzano questi modelli per creare le macchine necessarie al funzionamento dell'applicativo desiderato.
\subsection{Oracle Secure Global Desktop}
Oracle secure global desktop\footcite{site:osgd}  è uno strumento sviluppato da Oracle per l'accesso sicuro alle macchine virtuali da remoto.\\
Nel contesto aziendale viene utilizzato per centralizzare la definizione dei permessi di accesso alla flotta dei server del \gls{datacenter} privato. 
\subsection{Ambienti di sviluppo}
L'azienda non pone limitazioni per quanto riguarda gli \gls{ide}, tuttavia mostra una forte preferenza per i software sviluppati da JetBrains\footcite{site:jetbrains}, in particolare \textit{Intellij IDEA}, per Java, \textit{PyCharm} per Python e \textit{Webstorm} per Javascript. \\
Necessitando il mio progetto di diverse piattaforme d'esecuzione, ho optato per \textit{Visual Studio Code}\footcite{site:vscode}, \gls{ide} sviluppato da Microsoft\footcite{site:microsoft} e consigliato per lavorare per progetti multi linguaggio, in quanto estensibile tramite \gls{plugin} per supportare la maggior parte dei linguaggi di programmazione.  