% !TEX encoding = UTF-8
% !TEX TS-program = pdflatex
% !TEX root = ../tesi.tex

%**************************************************************
\chapter{Interessi e Aspettative}
\label{cap:processi-metodologie}
%**************************************************************

\intro{In questa sezione verranno presentati gli obiettivi del progetto, la pianificazione temporale del periodo di stage, gli interessi aziendali verso i progetti di stage e verso questo progetto nello specifico.\\
In conclusione verrà effettuata una valutazione delle mie aspettative personali.}\\

%**************************************************************
\section{Progetto Test di Carico}
\subsection{Descrizione}
\subsection{Obiettivi}
\subsection{Pianificazione Temporale}
\subsection{Risultati Attesi}
\subsection{Vincoli Metodologici}
\subsection{Vincoli Tecnologici}
\section{Interessi Aziendali}
\subsection{Stage nella strategia aziendale}
\subsection{I test di carico nel ciclo di vita del software}
Infocert tramite questo progetto formativo mira ad impostare le basi per l'inserimento in modo costante dei test di carico nel ciclo di vita del software da loro rilasciato. \\
Il team Legalmail già in passato ha effettuato test di questo tipo ma, a causa di stetti vincoli temporali e dall'impossibilità di posizionare risorse esclusivamente su questo progetto, non ha avuto modo di creare un'infrastruttura che rendesse questo processo ripetibile e semplice da implementare.  
Il gruppo infatti ha da tempo abbracciato la filosofia \gls{devops}, sviluppando \gls{framework} interni ottimizzati per aderire alla pratica dell'Integrazione Continua, proprio per questo motivo il team ha estremo interesse nello sviluppare un framework facilmente integrabile nella loro piattaforma anche per l'esecuzione dei test di carico. \\
Per l'azienda quindi, l'inserimento di una risorsa che si occupi di questo progetto a tempo pieno, risulta una scelta molto efficace. \\
L'esecuzione dei test di carico, inoltre, presenta vantaggi non indifferenti e ha valenza informativa non solo per l'area tecnica ma anche per gli \gls{stakeholders}: la prima infatti può analizzare i dati ottenuti per pianificare l'ampliamento o la riduzione della flotta dei server, mentre i secondi possono utilizzare i dati come certificazione della qualità del prodotto.


\section{Aspettative Personali}
