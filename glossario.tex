
%**************************************************************
% Acronimi
%**************************************************************
\renewcommand{\acronymname}{Acronimi e abbreviazioni}

\newacronym[description={\glslink{apig}{Application Program Interface}}]
{api}{API}{Application Program Interface}

\newacronym[description={\glslink{umlg}{Unified Modeling Language}}]
{uml}{UML}{Unified Modeling Language}
%**************************************************************
% Glossario
%**************************************************************
\renewcommand{\glossaryname}{Glossario}

\newglossaryentry{apig}
{
    name=\glslink{api}{API},
    text=Application Program Interface,
    sort=api,
    description={in informatica con il termine \emph{Application Programming Interface API} (ing. interfaccia di programmazione di un'applicazione) si indica ogni insieme di procedure disponibili al programmatore, di solito raggruppate a formare un set di strumenti specifici per l'espletamento di un determinato compito all'interno di un certo programma. La finalità è ottenere un'astrazione, di solito tra l'hardware e il programmatore o tra software a basso e quello ad alto livello semplificando così il lavoro di programmazione}
}
\newglossaryentry{spid}
{
	name=\glslink{spid}{SPID},
	text=SPID,
	sort=spid,
	description={SPID è il sistema di autenticazione che permette a cittadini ed imprese di accedere ai servizi online della pubblica amministrazione e dei privati aderenti con un’identità digitale unica. L’identità SPID è costituita da credenziali (nome utente e password) che vengono rilasciate all’utente e che permettono l’accesso a tutti i servizi online}
}
\newglossaryentry{agid}
{
	name=\glslink{agid}{AgID},
	text=AgID,
	sort=agid,
	description={L'Agenzia per l'Italia digitale (abbreviato AgID) è una agenzia pubblica italiana istituita dal governo Monti. L'Agenzia è sottoposta ai poteri di indirizzo e vigilanza del presidente del Consiglio dei ministri o del ministro da lui delegato. Svolge le funzioni ed i compiti ad essa attribuiti dalla legge al fine di perseguire il massimo livello di innovazione tecnologica nell'organizzazione e nello sviluppo della pubblica amministrazione e al servizio dei cittadini e delle imprese, nel rispetto dei principi di legalità, imparzialità e trasparenza e secondo criteri di efficienza, economicità ed efficacia}
}
\newglossaryentry{cna}
{
	name=\glslink{cna}{CNA},
	text=CNA,
	sort=cna,
	description={La CNA, Confederazione Nazionale dell'Artigianato e della Piccola e Media Impresa, dal 1946 rappresenta e tutela gli interessi delle micro, piccole e medie imprese, operanti nei settori della manifattura, costruzioni, servizi, trasporto, commercio e turismo, delle piccole e medie industrie, ed in generale del mondo dell’impresa e delle relative forme associate, con particolare riferimento al settore dell’artigianato; degli artigiani, del lavoro autonomo, dei professionisti  nelle sue diverse espressioni, delle imprenditrici e degli imprenditori e dei pensionati}
}
\newglossaryentry{devops}
{
	name=\glslink{devops}{DevOps},
	text=DevOps,
	sort=devops,
	description={DevOps (contrazione dei termini "development" e "operations") è una cultura/pratica dell'ingegneria del software che mira a unificare lo sviluppo del software e le operazioni effettuate per gestirlo. La caratteristica principale del movimento è il forte orientamento verso l'automazione e il monitoraggio di tutti gli step della costruzione del software, partendo dalla stesura della prima riga di codice fino alla gestione dell'infrastruttura}
}
\newglossaryentry{framework}
{
	name=\glslink{framework}{Framework},
	text=framework,
	sort=framework,
	description={Un framework, in informatica e specificatamente nello sviluppo software, è un'architettura logica di supporto (spesso un'implementazione logica di un particolare design pattern) su cui un software può essere progettato e realizzato, spesso facilitandone lo sviluppo da parte del programmatore}
}
\newglossaryentry{stakeholders}
{
	name=\glslink{stakeholders}{Stakeholders},
	text=stakeholders,
	sort=stakeholders,
	description={Tutti i soggetti, individui od organizzazioni, attivamente coinvolti in un’iniziativa economica (progetto, azienda), il cui interesse è negativamente o positivamente influenzato dal risultato dell’esecuzione, o dall’andamento, dell’iniziativa e la cui azione o reazione a sua volta influenza le fasi o il completamento di un progetto o il destino di un’organizzazione}
}
\newglossaryentry{microservizi}
{
	name=\glslink{microservizi}{Architettura a microservizi},
	text=microservizi,
	sort=Architettura a microservizi,
	description={L'architettura a microservizi è uno stile architetturale per lo sviluppo di una singola applicazione come un insieme di microservizi, questi sono dei servizi piccoli e autonomi, eseguiti come	processi distinti, che lavorano insieme comunicando mediante meccanismi leggeri. Ogni microservizio si occupa di una sola specifica unità applicativa.}
}
%TODO: inserire definizioni corrette
\newglossaryentry{way-of-working}
{
	name=\glslink{way-of-working}{Way Of Working},
	text=way of working,
	sort=way of working,
	description={L'architettura a microservizi è uno stile architetturale per lo sviluppo di una singola applicazione come un insieme di microservizi, questi sono dei servizi piccoli e autonomi, eseguiti come	processi distinti, che lavorano insieme comunicando mediante meccanismi leggeri. Ogni microservizio si occupa di una sola specifica unità applicativa.}
}
\newglossaryentry{integrazione-continua}
{
	name=\glslink{integrazione-continua}{CI},
	text=Integrazione Continua,
	sort=integrazione continua,
	description={L'architettura a microservizi è uno stile architetturale per lo sviluppo di una singola applicazione come un insieme di microservizi, questi sono dei servizi piccoli e autonomi, eseguiti come	processi distinti, che lavorano insieme comunicando mediante meccanismi leggeri. Ogni microservizio si occupa di una sola specifica unità applicativa.}
}
\newglossaryentry{deploy}
{
	name=\glslink{deploy}{Deploy},
	text=deploy,
	sort=deploy,
	description={L'architettura a microservizi è uno stile architetturale per lo sviluppo di una singola applicazione come un insieme di microservizi, questi sono dei servizi piccoli e autonomi, eseguiti come	processi distinti, che lavorano insieme comunicando mediante meccanismi leggeri. Ogni microservizio si occupa di una sola specifica unità applicativa.}
}
\newglossaryentry{branch}
{
	name=\glslink{branch}{Branch},
	text=branch,
	sort=branch,
	description={L'architettura a microservizi è uno stile architetturale per lo sviluppo di una singola applicazione come un insieme di microservizi, questi sono dei servizi piccoli e autonomi, eseguiti come	processi distinti, che lavorano insieme comunicando mediante meccanismi leggeri. Ogni microservizio si occupa di una sola specifica unità applicativa.}
}
\newglossaryentry{cfg-mgmt}
{
	name=\glslink{cfg-mgmt}{Configuration Management},
	text=configuration management,
	sort=configuration management,
	description={L'architettura a microservizi è uno stile architetturale per lo sviluppo di una singola applicazione come un insieme di microservizi, questi sono dei servizi piccoli e autonomi, eseguiti come	processi distinti, che lavorano insieme comunicando mediante meccanismi leggeri. Ogni microservizio si occupa di una sola specifica unità applicativa.}
}
\newglossaryentry{build}
{
	name=\glslink{build}{Build},
	text=build,
	sort=build,
	description={L'architettura a microservizi è uno stile architetturale per lo sviluppo di una singola applicazione come un insieme di microservizi, questi sono dei servizi piccoli e autonomi, eseguiti come	processi distinti, che lavorano insieme comunicando mediante meccanismi leggeri. Ogni microservizio si occupa di una sola specifica unità applicativa.}
}
\newglossaryentry{datacenter}
{
	name=\glslink{datacenter}{Datacenter},
	text=datacenter,
	sort=datacenter,
	description={L'architettura a microservizi è uno stile architetturale per lo sviluppo di una singola applicazione come un insieme di microservizi, questi sono dei servizi piccoli e autonomi, eseguiti come	processi distinti, che lavorano insieme comunicando mediante meccanismi leggeri. Ogni microservizio si occupa di una sola specifica unità applicativa.}
}
\newglossaryentry{release}
{
	name=\glslink{release}{Release},
	text=release,
	sort=release,
	description={L'architettura a microservizi è uno stile architetturale per lo sviluppo di una singola applicazione come un insieme di microservizi, questi sono dei servizi piccoli e autonomi, eseguiti come	processi distinti, che lavorano insieme comunicando mediante meccanismi leggeri. Ogni microservizio si occupa di una sola specifica unità applicativa.}
}
\newglossaryentry{sla}
{
	name=\glslink{sla}{Service Level Agreement (SLA)},
	text=SLA,
	sort=Service Level Agreement,
	description={I service level agreement (in italiano: accordo sul livello del servizio) sono strumenti contrattuali attraverso i quali si definiscono le metriche di servizio (es. qualità di servizio) che devono essere rispettate da un fornitore di servizi (provider) nei confronti dei propri clienti/utenti. Di fatto, una volta stipulato il contratto, assumono il significato di obblighi contrattuali.}
}
\newglossaryentry{ide}
{
	name=\glslink{ide}{Integrated Development Environment (IDE)},
	text=IDE,
	sort=Integrated Development Environment,
	description={I service level agreement (in italiano: accordo sul livello del servizio) sono strumenti contrattuali attraverso i quali si definiscono le metriche di servizio (es. qualità di servizio) che devono essere rispettate da un fornitore di servizi (provider) nei confronti dei propri clienti/utenti. Di fatto, una volta stipulato il contratto, assumono il significato di obblighi contrattuali.}
}
\newglossaryentry{plugin}
{
	name=\glslink{plugin}{plugin},
	text=plugin,
	sort=plugin,
	description={I service level agreement (in italiano: accordo sul livello del servizio) sono strumenti contrattuali attraverso i quali si definiscono le metriche di servizio (es. qualità di servizio) che devono essere rispettate da un fornitore di servizi (provider) nei confronti dei propri clienti/utenti. Di fatto, una volta stipulato il contratto, assumono il significato di obblighi contrattuali.}
}
\newglossaryentry{csr}
{
	name=\glslink{csr}{Corporate Social Responsability},
	text=Corporate Social Responsability,
	sort=Corporate Social Responsability,
	description={I service level agreement (in italiano: accordo sul livello del servizio) sono strumenti contrattuali attraverso i quali si definiscono le metriche di servizio (es. qualità di servizio) che devono essere rispettate da un fornitore di servizi (provider) nei confronti dei propri clienti/utenti. Di fatto, una volta stipulato il contratto, assumono il significato di obblighi contrattuali.}
}