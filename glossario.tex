
%**************************************************************
% Acronimi
%**************************************************************
\renewcommand{\acronymname}{Acronimi e abbreviazioni}

\newacronym[description={\glslink{apig}{Application Program Interface}}]
{api}{API}{Application Program Interface}

\newacronym[description={\glslink{umlg}{Unified Modeling Language}}]
{uml}{UML}{Unified Modeling Language}
%**************************************************************
% Glossario
%**************************************************************
\renewcommand{\glossaryname}{Glossario}

\newglossaryentry{apig}
{
    name=\glslink{api}{Application Program Interface},
    text=API,
    sort=API,
    description={in informatica con il termine \emph{Application Programming Interface API} (ing. interfaccia di programmazione di un'applicazione) si indica ogni insieme di procedure disponibili al programmatore, di solito raggruppate a formare un set di strumenti specifici per l'espletamento di un determinato compito all'interno di un certo programma. La finalità è ottenere un'astrazione, di solito tra l'hardware e il programmatore o tra software a basso e quello ad alto livello semplificando così il lavoro di programmazione}
}
\newglossaryentry{spid}
{
	name=\glslink{spid}{SPID},
	text=SPID,
	sort=spid,
	description={SPID è il sistema di autenticazione che permette a cittadini ed imprese di accedere ai servizi online della pubblica amministrazione e dei privati aderenti con un’identità digitale unica. L’identità SPID è costituita da credenziali (nome utente e password) che vengono rilasciate all’utente e che permettono l’accesso a tutti i servizi online}
}
\newglossaryentry{agid}
{
	name=\glslink{agid}{AgID},
	text=AgID,
	sort=agid,
	description={L'Agenzia per l'Italia digitale (abbreviato AgID) è una agenzia pubblica italiana istituita dal governo Monti. L'Agenzia è sottoposta ai poteri di indirizzo e vigilanza del presidente del Consiglio dei ministri o del ministro da lui delegato. Svolge le funzioni ed i compiti ad essa attribuiti dalla legge al fine di perseguire il massimo livello di innovazione tecnologica nell'organizzazione e nello sviluppo della pubblica amministrazione e al servizio dei cittadini e delle imprese, nel rispetto dei principi di legalità, imparzialità e trasparenza e secondo criteri di efficienza, economicità ed efficacia}
}
\newglossaryentry{cna}
{
	name=\glslink{cna}{CNA},
	text=CNA,
	sort=cna,
	description={La CNA, Confederazione Nazionale dell'Artigianato e della Piccola e Media Impresa, dal 1946 rappresenta e tutela gli interessi delle micro, piccole e medie imprese, operanti nei settori della manifattura, costruzioni, servizi, trasporto, commercio e turismo, delle piccole e medie industrie, ed in generale del mondo dell’impresa e delle relative forme associate, con particolare riferimento al settore dell’artigianato; degli artigiani, del lavoro autonomo, dei professionisti  nelle sue diverse espressioni, delle imprenditrici e degli imprenditori e dei pensionati}
}
\newglossaryentry{devops}
{
	name=\glslink{devops}{DevOps},
	text=DevOps,
	sort=devops,
	description={DevOps (contrazione dei termini "development" e "operations") è una cultura/pratica dell'ingegneria del software che mira a unificare lo sviluppo del software e le operazioni effettuate per gestirlo. La caratteristica principale del movimento è il forte orientamento verso l'automazione e il monitoraggio di tutti gli step della costruzione del software, partendo dalla stesura della prima riga di codice fino alla gestione dell'infrastruttura}
}
\newglossaryentry{framework}
{
	name=\glslink{framework}{Framework},
	text=framework,
	sort=framework,
	description={Un framework, in informatica e specificatamente nello sviluppo software, è un'architettura logica di supporto (spesso un'implementazione logica di un particolare design pattern) su cui un software può essere progettato e realizzato, spesso facilitandone lo sviluppo da parte del programmatore}
}
\newglossaryentry{stakeholders}
{
	name=\glslink{stakeholders}{Stakeholders},
	text=stakeholders,
	sort=stakeholders,
	description={Tutti i soggetti, individui od organizzazioni, attivamente coinvolti in un’iniziativa economica (progetto, azienda), il cui interesse è negativamente o positivamente influenzato dal risultato dell’esecuzione, o dall’andamento, dell’iniziativa e la cui azione o reazione a sua volta influenza le fasi o il completamento di un progetto o il destino di un’organizzazione}
}
\newglossaryentry{microservizi}
{
	name=\glslink{microservizi}{Architettura a microservizi},
	text=microservizi,
	sort=Architettura a microservizi,
	description={ L'architettura a microservizi è uno stile architetturale per lo sviluppo di una singola applicazione come un insieme di microservizi, questi sono dei servizi piccoli e autonomi, eseguiti come	processi distinti, che lavorano insieme comunicando mediante meccanismi leggeri. Ogni microservizio si occupa di una sola specifica unità applicativa}
}
%TODO: inserire definizioni corrette
\newglossaryentry{way-of-working}
{
	name=\glslink{way-of-working}{Way Of Working},
	text=way of working,
	sort=way of working,
	description={ Il way of working è l'insieme di metodi, strumenti e procedure atte a guidare e supportare il lavoro dei team aziendali. Questo modello per funzionare al meglio viene deciso a priori e viene adottato da tutto il team senza obiezioni. Questo non vuol dire che il WoW sia un entità costante, un buon Way of Working infatti deve essere in grado di evolversi e migliorare se stesso}
}
\newglossaryentry{integrazione-continua}
{
	name=\glslink{integrazione-continua}{CI},
	text=Integrazione Continua,
	sort=integrazione continua,
	description={ L'integrazione continua è una pratica dell'ignegneria del software atta a risolvere il problema dell'integration hell: l'insieme di problematiche collegate all'upgrade delle applicazioni in produzione. Facendo un riassunto la CI prevede che ogni commit inneschi un processo che configuri l'applicativo e lo verifichi, in modo da essere facilmente integrato negli ambienti di produzione. Questo processo è completamente automatizzato e prevede due varianti, la continuous delivery, che crea la versione rilasciabile dell'applicativo, e il continuous deploy, che rilascia automaticamente la release negli ambienti specificati}
}
\newglossaryentry{deploy}
{
	name=\glslink{deploy}{Deploy},
	text=deploy,
	sort=deploy,
	description={ Il deploy è l'ultimo step nel rilascio di una nuova versione del software e consiste nell'applicare le modifiche apportate al programma all'interno delle infrastrutture desiderate}
}
\newglossaryentry{branch}
{
	name=\glslink{branch}{Branch},
	text=branch,
	sort=branch,
	description={ Nell'ambito del controllo di versione, un branch è una duplicazione di un oggetto (codice sorgente, insieme di file, etc.) che permette di apportare modifiche allo stesso senza impattare la copia originale. I branch vengono solitamente effettuati per creare una nuova funzionalità o apporre una modifica per poi integrarla nel codice principale senza causare problemi nel frattempo, permettendo il lavoro di più persone sulla stessa codebase}
}
\newglossaryentry{cfg-mgmt}
{
	name=\glslink{cfg-mgmt}{Configuration Management},
	text=configuration management,
	sort=configuration management,
	description={ La gestione della configurazione è una pratica che si occupa di preparare il server ad accogliere l'applicativo desiderato, installando dipendenze, moduli e configurando l'ambiente d'esecuzione}
}
\newglossaryentry{build}
{
	name=\glslink{build}{Build},
	text=build,
	sort=build,
	description={ Nell'ingegneria del software una build è l'output ricavato dalla conversione dell'applicativo da codice sorgente a codice eseguibile}
}
\newglossaryentry{datacenter}
{
	name=\glslink{datacenter}{Datacenter},
	text=datacenter,
	sort=datacenter,
	description={ In parole semplici un datacenter è la sala macchine che ospita server, storage, gruppi di continuità e tutte le apparecchiature che consentono di governare i processi, le comunicazioni così come i servizi che supportano qualsiasi attività aziendale. Spesso alle aziende non conviene sostenere i costi di un datacenter privato, accedendo a datacenter di terzi tramite servizi di cloud}
}
\newglossaryentry{release}
{
	name=\glslink{release}{Release},
	text=release,
	sort=release,
	description={ Una release è l'output ottenuto dall'unione di build e configurazione. Una build infatti spesso, pur essendo eseguibile, non è pronta per essere utilizzata, in quanto priva della configurazione necessaria (url a database, credenziali, variabili d'ambiente, etc.) per fornire valore significativo. La release quindi è il software che viene effettivamente utilizzato negli ambienti esecutivi}
}
\newglossaryentry{sla}
{
	name=\glslink{sla}{Service Level Agreement (SLA)},
	text=SLA,
	sort=Service Level Agreement,
	description={ I service level agreement (in italiano: accordo sul livello del servizio) sono strumenti contrattuali attraverso i quali si definiscono le metriche di servizio (es. qualità di servizio) che devono essere rispettate da un fornitore di servizi (provider) nei confronti dei propri clienti/utenti. Di fatto, una volta stipulato il contratto, assumono il significato di obblighi contrattuali}
}
\newglossaryentry{ide}
{
	name=\glslink{ide}{Integrated Development Environment (IDE)},
	text=IDE,
	sort=Integrated Development Environment,
	description={Gli IDE (in italiano: ambienti di sviluppo integrato) sono strumenti software che supportano il programmatore nello sviluppo del codice sorgente. Questi strumenti solitamente offrono il verificatore di sintassi, l'auto completamento delle sentenze e gli ambienti di debug}
}
\newglossaryentry{plugin}
{
	name=\glslink{plugin}{plugin},
	text=plugin,
	sort=plugin,
	description={Il plugin in campo informatico è un programma non autonomo che interagisce con un altro programma per ampliarne o estenderne le funzionalità originarie}
}
\newglossaryentry{csr}
{
	name=\glslink{csr}{Corporate Social Responsability},
	text=Corporate Social Responsability,
	sort=Corporate Social Responsability,
	description={La CSR (in italiano: Responsabilità sociale d'impresa) è, nel gergo economico e finanziario, l'ambito riguardante le implicazioni di natura etica all'interno della visione strategica d'impresa: è una manifestazione della volontà delle grandi, piccole e medie imprese di gestire efficacemente le problematiche d'impatto sociale ed etico al loro interno e nelle zone di attività}
}
\newglossaryentry{debito-tecnico}
{
	name=\glslink{debito-tecnico}{Technical debt},
	text=debito tecnico,
	sort=Technical debt,
	description={ Il debito tecnico, nell'ambito dello sviluppo software, è un concetto che spiega il grosso costo da sostenere per modificare software figlio di scelte di design non appropriate. Capita spesso infatti che nuove funzionalità, vuoi per mancanza di tempo e/o di capacità, vengano sviluppate seguendo un approccio rapido piuttosto che ben congegnato, rendendo molto difficile la sua modifica in periodi successivi. Questo debito, se non ripagato subito, applica gli "interessi" nel tempo: ogni estensione del software fara affidamento sulle funzionalità "indebitate" rendendo molto costoso effettuare una modifica. Un esempio di debito tecnico si trova quando nel modificare una piccola parte del software, diventa necessario modificare tutte le parti che interagiscono con questa}
}
\newglossaryentry{qa}
{
	name=\glslink{qa}{Quality Assurance},
	text=Quality Assurance,
	sort=Quality Assurance,
	description={ Nell'ingegneria del software, la Quality Assurance è il processo che si occupa di verificare e validare il prodotto sviluppato}
}
\newglossaryentry{version-control}
{
	name=\glslink{version-control}{Version Control System},
	text=Version Control System,
	sort=Version Control System,
	description={ Nell'ingegneria del software, la Quality Assurance è il processo che si occupa di verificare e validare il prodotto sviluppato}
}
\newglossaryentry{cli}
{
	name=\glslink{cli}{Command Line Interface},
	text=CLI,
	sort=Command Line Interface,
	description={ Nell'ingegneria del software, la Quality Assurance è il processo che si occupa di verificare e validare il prodotto sviluppato}
}
\newglossaryentry{gui}
{
	name=\glslink{gui}{Graphical User Interface},
	text=GUI,
	sort=Graphical User Interface,
	description={ Nell'ingegneria del software, la Quality Assurance è il processo che si occupa di verificare e validare il prodotto sviluppato}
}
\newglossaryentry{dsl}
{
	name=\glslink{dsl}{Domain Specific Language},
	text=DSL,
	sort=Domain Specific Language,
	description={ Nell'ingegneria del software, la Quality Assurance è il processo che si occupa di verificare e validare il prodotto sviluppato}
}
\newglossaryentry{out-of-the-box}
{
	name=\glslink{out-of-the-box}{Out of the Box},
	text=Out of the box,
	sort=Out of the box,
	description={ Nell'ingegneria del software, la Quality Assurance è il processo che si occupa di verificare e validare il prodotto sviluppato}
}
\newglossaryentry{cloud}
{
	name=\glslink{cloud}{Cloud Computing},
	text=Cloud,
	sort=Cloud Computing,
	description={Il Cloud Computing è un approccio moderno alla distribuzione dei servizi e alla gestione dell’infrastruttura. Sostanzialmente si tratta di affidare la gestione fisica dell’infrastruttura a piattaforme specializzate che la erogano come servizio.}
}
\newglossaryentry{container-system}
{
	name=\glslink{container-system}{Container System},
	text=Container System,
	sort=Container System,
	description={Il Cloud Computing è un approccio moderno alla distribuzione dei servizi e alla gestione dell’infrastruttura. Sostanzialmente si tratta di affidare la gestione fisica dell’infrastruttura a piattaforme specializzate che la erogano come servizio.}
}
\newglossaryentry{intranet}
{
	name=\glslink{intranet}{Intranet},
	text=Intranet,
	sort=Intranet,
	description={In informatica e telecomunicazioni l'intranet è una rete aziendale privata, spesso completamente isolata dalla rete esterna (internet) o abilitata alla comunicazione verso l'esterno solo per determinati servizi.}
}
\newglossaryentry{refactor}
{
	name=\glslink{refactor}{Refactor},
	text=refactor,
	sort=Refactor,
	description={In informatica e telecomunicazioni l'intranet è una rete aziendale privata, spesso completamente isolata dalla rete esterna (internet) o abilitata alla comunicazione verso l'esterno solo per determinati servizi.}
}
\newglossaryentry{duck-typing}
{
	name=\glslink{duck-typing}{Duck Typing},
	text=Duck Typing,
	sort=Duck Typing,
	description={In informatica e telecomunicazioni l'intranet è una rete aziendale privata, spesso completamente isolata dalla rete esterna (internet) o abilitata alla comunicazione verso l'esterno solo per determinati servizi.}
}