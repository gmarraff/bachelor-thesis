
%**************************************************************
% Acronimi
%**************************************************************
\begin{comment}
\renewcommand{\acronymname}{Acronimi e abbreviazioni}

\newacronym[description={\glslink{apig}{Application Program Interface}}]
{api}{API}{Application Program Interface}

\newacronymdescription={\glslink{umlg}{Unified Modeling Language}}]
{uml}{UML}{Unified Modeling Language}
\end{comment}
%**************************************************************
% Glossario
%**************************************************************
\renewcommand{\glossaryname}{Glossario}

\newglossaryentry{apig}
{
    name=Application Program Interface,
    text=API,
    sort=Application Program Interface,
    description={in informatica con il termine Application Programming Interface (API) si indica ogni insieme di procedure disponibili al programmatore, di solito raggruppate a formare un set di strumenti specifici per l'espletamento di un determinato compito all'interno di un certo programma. La finalità è ottenere un'astrazione, di solito tra l'hardware e il programmatore o tra software a basso e quello ad alto livello semplificando così il lavoro di programmazione}
}
\newglossaryentry{spid}
{
	name=SPID,
	text=SPID,
	sort=spid,
	description={SPID è il sistema di autenticazione che permette a cittadini ed imprese di accedere ai servizi online della pubblica amministrazione e dei privati aderenti con un’identità digitale unica. L’identità SPID è costituita da credenziali (nome utente e password) che vengono rilasciate all’utente e che permettono l’accesso a tutti i servizi online}
}
\newglossaryentry{agid}
{
	name=AgID,
	text=AgID,
	sort=agid,
	description={L'Agenzia per l'Italia digitale (abbreviato AgID) è una agenzia pubblica italiana istituita dal governo Monti. L'Agenzia è sottoposta ai poteri di indirizzo e vigilanza del presidente del Consiglio dei ministri o del ministro da lui delegato. Svolge le funzioni ed i compiti ad essa attribuiti dalla legge al fine di perseguire il massimo livello di innovazione tecnologica nell'organizzazione e nello sviluppo della pubblica amministrazione e al servizio dei cittadini e delle imprese, nel rispetto dei principi di legalità, imparzialità e trasparenza e secondo criteri di efficienza, economicità ed efficacia}
}
\newglossaryentry{cna}
{
	name=CNA,
	text=CNA,
	sort=cna,
	description={La CNA, Confederazione Nazionale dell'Artigianato e della Piccola e Media Impresa, dal 1946 rappresenta e tutela gli interessi delle micro, piccole e medie imprese, operanti nei settori della manifattura, costruzioni, servizi, trasporto, commercio e turismo, delle piccole e medie industrie, ed in generale del mondo dell’impresa e delle relative forme associate, con particolare riferimento al settore dell’artigianato; degli artigiani, del lavoro autonomo, dei professionisti  nelle sue diverse espressioni, delle imprenditrici e degli imprenditori e dei pensionati}
}
\newglossaryentry{devops}
{
	name=DevOps,
	text=DevOps,
	sort=devops,
	description={DevOps (contrazione dei termini "development" e "operations") è una cultura/pratica dell'ingegneria del software che mira a unificare lo sviluppo del software e le operazioni effettuate per gestirlo. La caratteristica principale del movimento è il forte orientamento verso l'automazione e il monitoraggio di tutti gli step della costruzione del software, partendo dalla stesura della prima riga di codice fino alla gestione dell'infrastruttura}
}
\newglossaryentry{framework}
{
	name=Framework,
	text=framework,
	sort=framework,
	description={Un framework, in informatica e specificatamente nello sviluppo software, è un'architettura logica di supporto (spesso un'implementazione logica di un particolare design pattern) su cui un software può essere progettato e realizzato, spesso facilitandone lo sviluppo da parte del programmatore}
}
\newglossaryentry{stakeholders}
{
	name=Stakeholders,
	text=stakeholders,
	sort=stakeholders,
	description={Tutti i soggetti, individui od organizzazioni, attivamente coinvolti in un’iniziativa economica (progetto, azienda), il cui interesse è negativamente o positivamente influenzato dal risultato dell’esecuzione, o dall’andamento, dell’iniziativa e la cui azione o reazione a sua volta influenza le fasi o il completamento di un progetto o il destino di un’organizzazione}
}
\newglossaryentry{microservizi}
{
	name=Architettura a microservizi,
	text=microservizi,
	sort=Architettura a microservizi,
	description={ L'architettura a microservizi è uno stile architetturale per lo sviluppo di una singola applicazione come un insieme di microservizi, questi sono dei servizi piccoli e autonomi, eseguiti come	processi distinti, che lavorano insieme comunicando mediante meccanismi leggeri. Ogni microservizio si occupa di una sola specifica unità applicativa}
}
%TODO: inserire definizioni corrette
\newglossaryentry{way-of-working}
{
	name=Way Of Working,
	text=way of working,
	sort=way of working,
	description={ Il way of working (WoW) è l'insieme di metodi, strumenti e procedure atte a guidare e supportare il lavoro dei team aziendali. Questo modello per funzionare al meglio viene deciso a priori e viene adottato da tutto il team senza obiezioni. Questo non vuol dire che il WoW sia un entità costante, un buon Way of Working infatti deve essere in grado di evolversi e migliorare se stesso}
}
\newglossaryentry{integrazione-continua}
{
	name=Integrazione Continua,
	text=Integrazione Continua,
	sort=integrazione continua,
	description={ L'integrazione continua è una pratica dell'ignegneria del software atta a risolvere il problema dell'integration hell: l'insieme di problematiche collegate all'upgrade delle applicazioni in produzione. Facendo un riassunto la CI prevede che ogni commit inneschi un processo che configuri l'applicativo e lo verifichi, in modo da essere facilmente integrato negli ambienti di produzione. Questo processo è completamente automatizzato e prevede due varianti, la continuous delivery, che crea la versione rilasciabile dell'applicativo, e il continuous deploy, che rilascia automaticamente la release negli ambienti specificati}
}
\newglossaryentry{deploy}
{
	name=Deploy,
	text=deploy,
	sort=deploy,
	description={ Il deploy è l'ultimo step nel rilascio di una nuova versione del software e consiste nell'applicare le modifiche apportate al programma all'interno delle infrastrutture desiderate}
}
\newglossaryentry{branch}
{
	name=Branch,
	text=branch,
	sort=branch,
	description={ Nell'ambito del controllo di versione, un branch è una duplicazione di un oggetto (codice sorgente, insieme di file, etc.) che permette di apportare modifiche allo stesso senza impattare la copia originale. I branch vengono solitamente effettuati per creare una nuova funzionalità o apporre una modifica per poi integrarla nel codice principale senza causare problemi nel frattempo, permettendo il lavoro di più persone sulla stessa codebase}
}
\newglossaryentry{cfg-mgmt}
{
	name=Configuration Management,
	text=configuration management,
	sort=configuration management,
	description={ La gestione della configurazione è una pratica che si occupa di preparare il server ad accogliere l'applicativo desiderato, installando dipendenze, moduli e configurando l'ambiente d'esecuzione}
}
\newglossaryentry{build}
{
	name=Build,
	text=build,
	sort=build,
	description={ Nell'ingegneria del software una build è l'output ricavato dalla conversione dell'applicativo da codice sorgente a codice eseguibile}
}
\newglossaryentry{datacenter}
{
	name=Datacenter,
	text=datacenter,
	sort=datacenter,
	description={ In parole semplici un datacenter è la sala macchine che ospita server, storage, gruppi di continuità e tutte le apparecchiature che consentono di governare i processi, le comunicazioni così come i servizi che supportano qualsiasi attività aziendale. Spesso alle aziende non conviene sostenere i costi di un datacenter privato, accedendo a datacenter di terzi tramite servizi di cloud}
}
\newglossaryentry{release}
{
	name=Release,
	text=release,
	sort=release,
	description={ Una release è l'output ottenuto dall'unione di build e configurazione. Una build infatti spesso, pur essendo eseguibile, non è pronta per essere utilizzata, in quanto priva della configurazione necessaria (url a database, credenziali, variabili d'ambiente, etc.) per fornire valore significativo. La release quindi è il software che viene effettivamente utilizzato negli ambienti esecutivi}
}
\newglossaryentry{sla}
{
	name=Service Level Agreement,
	text=SLA,
	sort=Service Level Agreement,
	description={ I Service Level Agreement (SLA) sono strumenti contrattuali attraverso i quali si definiscono le metriche di servizio (es. qualità di servizio) che devono essere rispettate da un fornitore di servizi (provider) nei confronti dei propri clienti/utenti. Di fatto, una volta stipulato il contratto, assumono il significato di obblighi contrattuali}
}
\newglossaryentry{ide}
{
	name=Integrated Development Environment,
	text=IDE,
	sort=Integrated Development Environment,
	description={Gli Integrated Development Environment (IDE) sono strumenti software che supportano il programmatore nello sviluppo del codice sorgente. Questi strumenti solitamente offrono il verificatore di sintassi, l'auto completamento delle sentenze e gli ambienti di debug}
}
\newglossaryentry{plugin}
{
	name=Plugin,
	text=plugin,
	sort=plugin,
	description={Il plugin in campo informatico è un programma non autonomo che interagisce con un altro programma per ampliarne o estenderne le funzionalità originarie}
}
\newglossaryentry{csr}
{
	name=Corporate Social Responsability,
	text=Corporate Social Responsability,
	sort=Corporate Social Responsability,
	description={La Corporate Social Responsability (CSR) è, nel gergo economico e finanziario, l'ambito riguardante le implicazioni di natura etica all'interno della visione strategica d'impresa: è una manifestazione della volontà delle grandi, piccole e medie imprese di gestire efficacemente le problematiche d'impatto sociale ed etico al loro interno e nelle zone di attività}
}
\newglossaryentry{debito-tecnico}
{
	name=Technical debt,
	text=debito tecnico,
	sort=Technical debt,
	description={ Il debito tecnico, nell'ambito dello sviluppo software, è un concetto che spiega il grosso costo da sostenere per modificare software figlio di scelte di design non appropriate. Capita spesso infatti che nuove funzionalità, vuoi per mancanza di tempo e/o di capacità, vengano sviluppate seguendo un approccio rapido piuttosto che ben congegnato, rendendo molto difficile la sua modifica in periodi successivi. Questo debito, se non ripagato subito, applica gli "interessi" nel tempo: ogni estensione del software fara affidamento sulle funzionalità "indebitate" rendendo molto costoso effettuare una modifica. Un esempio di debito tecnico si trova quando nel modificare una piccola parte del software, diventa necessario modificare tutte le parti che interagiscono con questa}
}
\newglossaryentry{qa}
{
	name=Quality Assurance,
	text=Quality Assurance,
	sort=Quality Assurance,
	description={ Nell'ingegneria del software, la Quality Assurance (QA) è il processo che si occupa di verificare e validare il prodotto sviluppato}
}
\newglossaryentry{version-control}
{
	name=Version Control System,
	text=Version Control System,
	sort=Version Control System,
	description={ Il controllo versione (versioning), in informatica, è la gestione di versioni multiple di un insieme di informazioni. Gli strumenti software per il controllo versione sono chiamati VCS, Version Control System}
}
\newglossaryentry{cli}
{
	name=Command Line Interface,
	text=CLI,
	sort=Command Line Interface,
	description={ La Command Line Interface (CLI) in informatica indica una tipologia di interfaccia utente caratterizzata da un'interazione di tipo testuale tra utente ed elaboratore. L'utente impartisce comandi testuali in input mediante tastiera alfanumerica e riceve risposte testuali in output dall'elaboratore mediante display o stampante alfanumerici}
}
\newglossaryentry{gui}
{
	name=Graphical User Interface,
	text=GUI,
	sort=Graphical User Interface,
	description={ L'interfaccia grafica (nota anche come GUI (dall'inglese Graphical User Interface), in informatica è un tipo di interfaccia utente che consente l'interazione uomo-macchina in modo visuale utilizzando rappresentazioni grafiche piuttosto che utilizzando una interfaccia a riga di comando (CLI)}
}
\newglossaryentry{dsl}
{
	name=Domain Specific Language,
	text=DSL,
	sort=Domain Specific Language,
	description={ Un Domain Specific Language (DSL) è un linguaggio di programmazione specializzato nell'assolvimento di un determinato compito, rendendolo inadeguato per uso generale. I DSL possono essere ricavati da linguaggi di programmazione a scopo generale (GPL) per risolvere determinati problemi. Un esempio di DSL è l'HTML, linguaggio di markup ricavato da XML (usato per descrivere documenti), che si specializza nel descrivere pagine web}
}
\newglossaryentry{out-of-the-box}
{
	name=Out of the Box,
	text=Out of the box,
	sort=Out of the box,
	description={ Nei prodotti software, una funzionalità è definita Out of the box (OOTB) quando è presente e funzionante nell'applicativo senza dover effettuare installazioni o configurazioni aggiuntive}
}
\newglossaryentry{cloud}
{
	name=Cloud Computing,
	text=Cloud,
	sort=Cloud Computing,
	description={Il Cloud Computing è un approccio moderno alla distribuzione dei servizi e alla gestione dell’infrastruttura. Sostanzialmente si tratta di affidare la gestione fisica dell’infrastruttura a piattaforme specializzate che la erogano come servizio}
}
\newglossaryentry{container-system}
{
	name=Container System,
	text=Container System,
	sort=Container System,
	description={Un Container System è un sistema di virtualizzazione dove un sistema operativo permette l'istanziazione di più spazi indipendenti contenenti anch'essi sistemi operativi. Questi spazi sono visti dal programma eseguito su di essi come computer veri e propri, contenenti tutte le funzionalità che una macchina fisica può offrire. In particolare questi container vengono utilizzati per migliorare la distribuzione del software, proprio come dei container fisici infatti questi container virtuali possono essere "montati" su qualsiasi tipo di server con facilità}
}
\newglossaryentry{intranet}
{
	name=Intranet,
	text=Intranet,
	sort=Intranet,
	description={In informatica e telecomunicazioni l'intranet è una rete aziendale privata, spesso completamente isolata dalla rete esterna (internet) o abilitata alla comunicazione verso l'esterno solo per determinati servizi}
}
\newglossaryentry{refactor}
{
	name=Refactoring,
	text=refactoring,
	sort=Refactoring,
	description={Nell'ingegneria del software, il Refactoring è una tecnica utilizzata per modificare il codice, migliorandone leggibilità, usabilità, manutenibilità e riusabilità, senza variarne il comportamento esterno. Il Refactoring viene fatto a posteriori, ovvero su codice già esistente}
}
\newglossaryentry{duck-typing}
{
	name=Duck Typing,
	text=Duck Typing,
	sort=Duck Typing,
	description={Nei linguaggi di programmazione orientati agli oggetti, il duck typing si riferisce ad uno stile di tipizzazione dinamica dove la semantica di un oggetto è determinata dall'insieme corrente dei suoi metodi e delle sue proprietà anziché dal fatto di estendere una particolare classe o implementare una specifica interfaccia}
}
\newglossaryentry{dinje}
{
	name=Dependency Injection,
	text=Dependency Injection,
	sort=Dependency Injection,
	description={Il Dependency Injection è un design pattern architetturale atto a separare il comportamento di un componente dalla risoluzione delle sue dipendenze. Il componente quindi non sarà più dipendente da classi concrete, ma da interfacce, le cui implementazioni verranno "iniettate" da componenti dedicati allo scopo, chiamati Injector. L'utilizzo di questo pattern migliora estensibilità e verificabilità (semplifica il Mocking) del codice}
}
\newglossaryentry{traceback}
{
	name=Traceback,
	text=traceback,
	sort=Traceback,
	description={In informatica un traceback (o stack trace) è l'elenco della pila di chiamate attive nel momento in cui il traceback viene invocato. Per esempio, durante il sollevamento dell'eccezione, molti linguaggi di programmazione (come Python) stampano la lista di tutto lo stack di funzioni chiamate per giungere a quell'errore}
}
\newglossaryentry{smtp}
{
	name=Simple Mail Transfer Protocol,
	text=SMTP,
	sort=Simple Mail Transfer Protocol,
	description={Il Simple Mail Transfer Protocol (SMTP) è un protocollo per la trasmissione di E-Mail. Al giorno d'oggi viene prevalentemente usato per l'invio ma non per la ricezione}
}
\newglossaryentry{scp}
{
	name=Secure Copy Protocol,
	text=SCP,
	sort=Secure Copy Protocol,
	description={Secure Copy Protocol (SCP) è un protocollo per la trasmissione di file tra host remoti, o locale-remoto, in modo sicuro, sfruttando il protocollo SSH}
}
\newglossaryentry{ssh}
{
	name=Secure SHell,
	text=SSH,
	sort=Secure SHell,
	description={Secure SHell (SSH) è un protocollo che garantisce una connessione sicura e cifrata tra due macchine (host) non direttamente collegate tra loro}
}
\newglossaryentry{yaml}
{
	name=YAML Ain't Markup Language,
	text=YAML,
	sort=YAML Ain't Markup Language,
	description={YAML, acronimo ricorsivo di YAML Ain't Markup Language, è un formato per la serializzazione dei dati pensato per essere facilmente leggibile dall'essere umano, oltre che dal computer}
}
\newglossaryentry{wrapper}
{
	name=Wrapper,
	text=wrapper,
	sort=wrapper,
	description={In informatica, un Wrapper è un modulo software che ne "riveste" un altro, ovvero che funziona da tramite fra i propri clienti (che usano l'interfaccia del wrapper) e il modulo rivestito (che svolge effettivamente i servizi richiesti, su delega dell'oggetto wrapper)}
}
\newglossaryentry{shell}
{
	name=Shell,
	text=Shell,
	sort=Shell,
	description={La shell (detta anche interprete dei comandi), in informatica, è la parte di un sistema operativo che permette agli utenti di interagire con il sistema stesso, impartendo comandi e richiedendo l'avvio di altri programmi. Insieme al kernel costituisce una delle componenti principali di un sistema operativo}
}
\newglossaryentry{snakecase}
{
	name=snake\_case,
	text=snake\_case,
	sort=snake\_case,
	description={Lo snake case, o snake\_case, è la pratica di scrivere gli identificatori separando le parole che li compongono tramite trattino basso (o underscore: \_). Solitamente viene utilizzato per descrivere variabili e identificatori di funzioni}
}
\newglossaryentry{camelcase}
{
	name=CamelCase,
	text=CamelCase,
	sort=Camel Case,
	description={Con il termine CamelCase si definisce la pratica di scrivere gli identificatori non separando le parole che lo compongono ma ponendo in maiuscolo la prima lettera delle stesse. Solitamente viene utilizzato per annotare i nomi delle classi (eg. MiaClasse)}
}
\newglossaryentry{filesystem}
{
	name=Filesystem,
	text=filesystem,
	sort=Filesystem,
	description={Un Filesystem, in informatica, indica informalmente un meccanismo con il quale i file sono posizionati e organizzati su dispositivi di archiviazione}
}
\newglossaryentry{crash}
{
	name=Crash di sistema,
	text=crash,
	sort=Crash di sistema,
	description={Il termina crash, in Informatica, indica il blocco o la terminazione improvvisa, non richiesta e inaspettata di un programma in esecuzione (sistema operativo o applicazione), oppure il blocco completo dell'intero computer}
}
\newglossaryentry{ca}
{
	name=Certification Authority,
	text=Certification Authority,
	sort=Certification Authority,
	description={In crittografia, una Certificate Authority (CA) è un soggetto terzo di fiducia, pubblico o privato, abilitato ad emettere un certificato digitale tramite una procedura di certificazione che segue standard internazionali e in conformità alla normativa europea e nazionale in materia}
}
\newglossaryentry{rest}
{
	name=REpresentational State Transfer,
	text=REST,
	sort=Representational State Transfer,
	description={Representational State Transfer è un tipo di architettura software per i sistemi distribuiti, si basa su HTTP, il funzionamento prevede una struttura degli URL ben definita e l’utilizzo dei verbi HTTP specifici per accesso/creazione/modifica/eliminazione alle risorse}
}
\newglossaryentry{agile}
{
	name=Metodologie Agile,
	text=Agile,
	sort=Agile,
	description={Nell'ingegneria del software, l'espressione metodologia agile (letto all'inglese) si riferisce a un insieme di metodi di sviluppo del software emersi a partire dai primi anni 2000 e fondati su un insieme di principi comuni, direttamente o indirettamente derivati dai principi del "Manifesto per lo sviluppo agile del software" pubblicato nel 2001 da Kent Beck, Robert C. Martin, Martin Fowler e altri. I metodi agile si contrappongono al modello a cascata e altri modelli di sviluppo tradizionali, proponendo un approccio meno strutturato e focalizzato sull'obiettivo di consegnare al cliente, in tempi brevi e in modo frequente, software funzionante e di qualità}
}
\newglossaryentry{task}
{
	name=Task,
	text=task,
	sort=Task,
	description={Nel gergo lavorativo, un task (compito) è la più piccola unità di lavoro soggetta a responsabilità di gestione. Un task è un compito ben definito (spesso formalmente) assegnato generalmente ad una sola persona}
}
\newglossaryentry{rmi}
{
	name=Remote Method Invocation,
	text=RMI,
	sort=Remote Method Invocation,
	description={In informatica, e in particolare nel contesto del linguaggio di programmazione orientato agli oggetti Java, Remote Method Invocation (invocazione remota di metodi) o RMI è una tecnologia che consente a processi Java distribuiti di comunicare attraverso una rete. I metodi, invocati localmente, vengono propagati alle destinazioni remote tramite una comunicazione di tipo client-server e sono completamente trasparenti all'utente}
}
\newglossaryentry{websocket}
{
	name=WebSocket,
	text=WebSocket,
	sort=WebSocket,
	description={WebSocket è una tecnologia web che fornisce canali di comunicazione full-duplex attraverso una singola connessione TCP. Viene spesso utilizzato per l'implementazione di applicazioni web a tempo reale, dove non è solo l'utente (client) a richiedere aggiornamenti al server, ma quest'ultimo è in grado di notificare al client eventuali novità}
}
\newglossaryentry{milestone}
{
	name=Milestone,
	text=milestone,
	sort=Milestone,
	description={Una milestone rappresenta un punto nel tempo con un significato strategico ad esso associato (punto di riferimento). Questa milestone è insignificante presa da sola, deve essere infatti associata ad un insieme di risultati che determinano lo stato di avanzamento di un prodotto/progetto in quel particolare momento nel tempo. La milestone ha valenza sia retroattiva che proattiva: ci dice che in quel momento del tempo abbiamo raggiunto (o dobbiamo raggiungere) quei risultati}
}