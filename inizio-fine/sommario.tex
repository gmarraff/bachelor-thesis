% !TEX encoding = UTF-8
% !TEX TS-program = pdflatex
% !TEX root = ../tesi.tex

%**************************************************************
% Sommario
%**************************************************************
\cleardoublepage
\phantomsection
\pdfbookmark{Sommario}{Sommario}
\begingroup
\let\clearpage\relax
\let\cleardoublepage\relax
\let\cleardoublepage\relax

\chapter*{Sommario}

Il presente documento descrive il lavoro svolto durante il periodo di stage, della durata trecentoventi ore, dal laureando Gianluca Marraffa presso l'azienda \myCompany. \\
Lo scopo ultimo dello stage si proponeva come l'individuazione di una infrastruttura ideale per l'esecuzione di test di carico. In primo luogo si chiedeva un'analisi dello stato dell'arte dei vari strumenti di Load Testing presenti sul mercato volta a sottolineare pro e contro delle varie soluzioni, successivamente si sarebbe scelta la soluzione più adatta alle caratteristiche aziendali e del prodotto in esame. \\
In secondo luogo si chiedeva di instanziare un'infrastruttura atta all'esecuzione dei test in ambiente di sviluppo e accettazione con la relativa esecuzione e analisi dei risultati prodotti. \\
Per ultimo veniva richiesta la stesura di un documento che dimostrasse le scelte effettuate e permettesse al team di proseguire con lo sviluppo dell'infrastruttura realizzata.

%\vfill
%
%\selectlanguage{english}
%\pdfbookmark{Abstract}{Abstract}
%\chapter*{Abstract}
%
%\selectlanguage{italian}

\endgroup			

\vfill

